% Gemini theme
% https://github.com/anishathalye/gemini

\documentclass[final]{beamer}

% ====================
% Packages
% ====================

\usepackage[T1]{fontenc}
\usepackage{relsize}
\usepackage{lmodern}
\usepackage[size=a0,orientation=portrait,scale=1.0]{beamerposter}
\usetheme{gemini}
\usecolortheme{mitmarch}
\usepackage{graphicx}
\usepackage{booktabs}
\usepackage{tikz}
\usepackage{pgfplots}
\usepackage{natbib}
\usepackage{paralist}
\usepackage{subcaption}
\usepackage{multicol}
\usepackage{todonotes}

% ====================
% Lengths
% ====================

% If you have N columns, choose \sepwidth and \colwidth such that
% (N+1)*\sepwidth + N*\colwidth = \paperwidth
\newlength{\sepwidth}
\newlength{\colwidth}
\setlength{\sepwidth}{0.025\paperwidth}
\setlength{\colwidth}{0.3\paperwidth}

\newcommand{\separatorcolumn}[1][\sepwidth]{\begin{column}{#1}\end{column}}
\newcommand{\hemph}[1]{{\color{mitred}#1}}

% ====================
% Title
% ====================

\title{
  Título
}
\subtitle{
  Subtítulo (si hay)
}

\author{Lucas Somacal \inst{1} \and Martin A. Miguel \inst{1,2}} 

\institute[shortinst]{\inst{1} Universidad de Buenos Aires. Facultad de
Ciencias Exactas y Naturales. Departamento de Computaci\'on. Buenos Aires,
Argentina \samelineand \inst{2} CONICET-Universidad de Buenos Aires. Instituto
de Investigaci\'on en Ciencias de la Computaci\'on (ICC). Buenos Aires,
Argentina \inst{3} Laboratorio de Neurociencia Integrativa. Universidad Torcuato Di Tella.
    Buenos Aires, Argentina.}

% ====================
% Body
% ====================

\begin{document}
\begin{frame}[t]
    \begin{columns}
    \separatorcolumn[0.02\textwidth]
    \begin{column}{0.02\textwidth}
        %\hspace{-1.125cm}
        {\color{darkgray} \huge \rotatebox{90}{\textbf{\parbox{10cm}{
            \centering 
            \emph{TL}; DR}}}}
    \end{column}
    \separatorcolumn[0.02\textwidth]
    \begin{column}{0.52\textwidth}
        \vspace{1.5cm}
        {\Huge

        \centering

      \emph{Resumen} super corto

        }
        \vspace{2cm}
    \end{column}
    \separatorcolumn[0.02\textwidth]
    \begin{column}{0.4\textwidth}
        {\large
        \vspace{0.3cm}
        \begin{table}
          \caption*{\large FIGURA PRINCIPAL}
          \missingfigure{FIGURA PRINCIPAL}
        \end{table}
        }
    \end{column}
    \separatorcolumn[0.01\textwidth]
    \end{columns}
    %% Row: TL; DR %%
    \begin{columns}[T]
    \separatorcolumn[0.02\textwidth]
        \begin{column}{0.02\textwidth}
        \vspace{12cm} %\hspace{-1.125cm}
        {\color{darkgray} \huge \rotatebox{90}{\textbf{
            \centering 
            3' Speech
        }}
            }
    \end{column}
    \separatorcolumn[0.02\textwidth]
        \begin{column}{0.94\textwidth}
        \begin{column}{\textwidth}
            \vspace{2cm}
            \huge
            {\usebeamercolor[fg]{item} Grand goal:} reach a computational model
            to analyze affect in music.
            \vspace{2cm}
        \end{column}
        \begin{column}{0.30\textwidth}
        \begin{largeblock}{Anteriormente...}
            \begin{itemize}
            \setlength\itemsep{1em}
              \item Referencias previas

            \end{itemize}
        \end{largeblock}
    \end{column}
    \separatorcolumn[0.02\textwidth]
    \begin{column}{0.30\textwidth}
        \begin{largeblock}{Nuestra propuesta}
            \begin{itemize}
            \setlength\itemsep{1.2em}
              \item Propuestas
            \end{itemize}
        \end{largeblock}
    \end{column}
    \separatorcolumn[0.02\textwidth]
    \begin{column}{0.30\textwidth}
        \begin{largeblock}{Funcionó?}
            \begin{itemize}
            \setlength\itemsep{1em}
            \item Breves resultados
           \end{itemize}
        \end{largeblock}
    \end{column}
    \end{column}
    \separatorcolumn[0.01\textwidth]
    \end{columns}

    %% Row: The Results %%
    \vspace{-1cm}
    \begin{beamercolorbox}[wd=\pagewidth]{separator}
    %\end{beamercolorbox}
    \vspace{6pt}
    \begin{columns}[T]
    \separatorcolumn[0.02\textwidth]
    \begin{column}{0.02\textwidth}
        \vspace{3.25cm}
        {\color{darkgray} \huge \rotatebox{90}{\textbf{Resultados}}}
    \end{column}
    \separatorcolumn[0.02\textwidth]
    \begin{column}[t]{0.38\textwidth}
      THT's pulse clarity correlates significantly with various datasets.

      \vspace{0.7cm}
      \hspace{-0.5cm}%
      \missingfigure{Resultado A}
    \end{column}
    \begin{column}[t]{0.24\textwidth}
      THT's pulse clarity correlations are comparable with other models of
      pulse clarity.

      \missingfigure{Resultado B}

    \end{column}
        \separatorcolumn[0.01\textwidth]
        \begin{column}[t]{0.23\textwidth}
    THT's pulse clarity presents an inverted U-shaped relationship with
            musicality and need-to-move responses in the Rhythms dataset.
        \missingfigure{Resultado C}

    \end{column}
    \separatorcolumn[0.01\textwidth]
    \end{columns}
    \vspace{6pt}
    %\begin{beamercolorbox}[bg, ht=6pt]{separator}
    \end{beamercolorbox}

    %% Row: Details %%
    \begin{columns}[T]
    \separatorcolumn[0.02\textwidth]
    \begin{column}{0.02\textwidth}
        \vspace{14cm}
        {\color{darkgray} \huge \rotatebox{90}{\textbf{The Details}}}
    \end{column}
    \separatorcolumn[0.02\textwidth]
    \begin{column}{0.93\textwidth}
    \begin{column}{0.5\textwidth}
    \begin{block}{The Model}
        {\small
        THT is an agent-based model. Each agent keeps tabs on a tactus
        hypothesis and hence are named \emph{hypothesis trackers}. A possible
        tactus is represented by a phase ($\rho$) and a period ($\delta$).
        Trackers are created and updated while listening to the rhythm.
        Hypotheses have a \emph{certainty score} in [0, 1]. Hypotheses
        parameters and score evolve overtime.

        }

    \end{block}
    \begin{block}{\normalsize \emph{Future Work}}
      \begin{minipage}{0.37\textwidth}
      {\small
        The pulse clarity curve should be evaluated and calibrated on
        rhythmic stimuli that varies through time.
      }
      \vspace{1cm}
      \par{}
      {\footnotesize
        Figure presents example of the evaluation to-be performed,
        comparing pulse clarity with tapping variability in the MIREX
        dataset. 
      }
      \end{minipage}
      \begin{minipage}{0.62\textwidth}
      \raggedleft
      \end{minipage}
    \end{block}
    \end{column}
    \separatorcolumn[0.02\textwidth]
    \begin{column}{0.472\textwidth}
        \begin{block}{The Datasets}
            {\small
            \begin{columns}
            \begin{column}{0.49\textwidth}
            
            \vspace{0.7em}
            \hemph{Rhythms:}
            \begin{itemize}
              \item {\usebeamercolor[fg]{item}Task:} tap the beat freely while
            listening to rhythmic examples of varied difficulty.
              \item 33 ~30-second rhythmic stimuli of varying complexity (5
                isochronous, 11 from \cite{fitch2007perception}, 7 from
                \cite{povel1985perception} and 10 new)
              \item 30 participants of varying musical training (mean 4.85 years,
                sd=3.90)
              \item Participants also reported tapping difficulty, musicality of
                the stimulus and the feeling of needing to move.
              \item Pulse clarity was estiimated as negative tapping difficulty
                (PC) and negative inter-tap interval entropy (ITI-E)                
            \end{itemize}
            \end{column}
            \begin{column}{0.49\textwidth}
              \hemph{MIREX: \nocite{website:mirex}}
                \begin{itemize}
                \item 20 30-second musical excerpts of various musical genres
                \item 40 annotators
                \item Participants tapped to a self selected beat. Pulse clarity was
                 estimated as the negative inter-tap interval entropy (ITI-E)                
               \end{itemize}
              \hemph{Soundtracks \citep{lartillot2008multi, eerola2020dataset}:} 
                \begin{itemize}
                  \item 100 5-minute excerpts of soundtracks
                  \item 25 musically trained annotators
                  \item Participants provided pulse clarity ratings (PC)
                \end{itemize}
            \end{column}
          \end{columns}
            }
        \end{block}
        \begin{block}{Comparison models}
          {\small
          \begin{itemize}
            \item \hemph{MIRToolbox \citep{lartillot2008matlab}:} estimates
              periodicities of a onset activation function obtained from the audio
              signal. Here we use MIRToolbox 1.7.2 (model 1).
            \item \hemph{MAIPC (under review):} provides various estimates of pulse
              clarity from the inner workings of a deep-learning beat tracking
              model. Here we use the entropy of the distribution of possible
              beats estimated in the model from \cite{krebs2015efficient}.
          \end{itemize}
          }
        \end{block}
        \begin{block}{\normalsize References}
{ \scriptsize
\bibliographystyle{abbrvnat}
\bibliography{references-index} % Use the example bibliography file sample.bib
}
        \end{block}
    \end{column}
    \end{column}
    \separatorcolumn[0.01\textwidth]
    \end{columns}

    %% Row: Else %%
%    \begin{columns}[T]
%    \separatorcolumn[0.01\textwidth]
%    \begin{column}{0.48\textwidth}
%        \begin{block}{\normalsize Abstract}
%        \small
%Music has a unique capability to evoke emotions. A very interesting one is that
%of tension. Tension arises in a situation of dissonance and uncertainty that
%yearns for resolution. This is a kind of affect that develops in time, given
%that it is dependent on expectations about what will happen next. Specifically
%in rhythms, two concepts have been explored that relate to the comfort and
%understanding of music: pulse clarity and rhythm complexity. Several
%computational models have been introduced to analyze these concepts. In most
%cases their analysis is static -- i.e. full passage is studied -- and does not
%consider how they evolve in time. We present THT, a novel beat tracking model,
%that given the onset times of a rhythmic passage provides continuous
%information of which tacti are most reasonable and how salient they are. It
%works by tracking multiple tactus hypothesis overtime and providing a score
%designed to reflect confidence on the tactus. In this work we set to evaluate
%the output of the THT model as a proxy for pulse clarity. The mean of the
%continuous tactus confidence curve was taken as the pulse clarity score of the
%model and we performed a beat tapping experiment to evaluate our metric. The
%experiment consisted of asking participants (N=28) to tap the subjective beat
%while listening to 33 rhythmic passages. After each trial they were asked about
%the task's difficulty as a subjective measure of clarity. We also calculated
%the within-subject tapping clarity as an empirical measurement. The proposed
%computational metric correlated significantly against both subjective and
%objective measures (spearman r < -0.71, p < 0.001). Comparison with other
%models yielded similar results (Lartillot 2008, Fitch 2007, Povel 1985). This
%positive result allows us to inspect music emotions that arise from changes in
%rhythm perception.
%        \end{block}
%    \end{column}
%    \separatorcolumn[0.03\textwidth]
%    \begin{column}{0.48\textwidth}
%        \nocite{fitch2007perception}
%        \nocite{lartillot2008multi}
%        \begin{block}{\normalsize References}
%{ \small
%\bibliographystyle{abbrvnat}
%\bibliography{poster} % Use the example bibliography file sample.bib
%}
%        \end{block}
%    \end{column}
%    \separatorcolumn[0.01\textwidth]
%    \end{columns}

\begin{beamercolorbox}[wd=\paperwidth, sep=.6cm]{headline}

\vspace{0.65cm}

\begin{columns}[c]
\separatorcolumn[0.01\textwidth]
\begin{column}{0.4\textwidth}
%\includegraphics[height=7.2cm]{imgs/poster-footer.pdf}
\end{column}
%\begin{column}[c]{0.59\textwidth}
%\raggedleft
%\includegraphics[height=4cm]{imgs/logo_icc.pdf}
%    \hspace{2cm}
%\includegraphics[height=4cm]{imgs/logo_liaa.pdf}
%\end{column}
\end{columns}

\vspace{0.65cm}

\end{beamercolorbox}
\end{frame}


\end{document}
