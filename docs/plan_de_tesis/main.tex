\documentclass[10pt]{article}
\usepackage{nopageno}
\usepackage[utf8]{inputenc}
\usepackage[spanish]{babel}
\usepackage[left=1in, right=1in, top=1in, bottom=1in]{geometry}
\usepackage{url}
\usepackage{xcolor}
\usepackage{todonotes}
\usepackage[hidelinks]{hyperref}
\usepackage[normalem]{ulem}

\newcommand{\suggestM}[2]{{\color{orange} #1}{ \color{red}\sout{#2}}}
\newcommand{\tripleSignature}[3]{
\begin{minipage}[c]{\textwidth}

\vspace{1.5cm}

\makebox[12cm][c]{
\qquad \qquad \qquad \qquad \qquad \makebox[5cm][c] {\hrulefill} \quad \makebox[5cm][c] {\hrulefill} \quad \makebox[5cm][c] {\hrulefill}
}
\makebox[14cm][c]{
\qquad \qquad \qquad #1 \qquad \qquad  #2 \qquad #3 
}
% \vspace{1cm}
\end{minipage}
}


\title{{\normalsize Propuesta de tesis} \\ \textbf{
Variational Autoencoders para el modelado de estilos de música: ¿cómo sonaría un ragtime
si lo hubiera escrito Bach?
}\todo[inline]{Definir}} 
\author{
  \centering
  \begin{tabular}{c c c}
    Alumno & Director & Co-director\\
    Lucas Somacal &  Martin Miguel & Diego Fernández Slezak \\
    lsomacal@gmail.com & mmiguel@dc.uba.ar & dfslezak@dc.uba.ar \\
    LU: 249/16 & & \\
  \end{tabular}
}

\date{}

\begin{document}

% título tentativo de la tesis, introducción o antecedentes en la temática, actividades y metodología a realizar por el tesista, factibilidad de realización en el plazo establecido de común acuerdo entre el director y el tesista y finalmente referencias bibliográficas. La propuesta debe tener como máximo 3 carillas sin contar la bibliografía.

\maketitle

\section*{Introducción}
En los últimos años, nuevas propuestas de arquitecturas de redes neuronales han
permitido modelar producciones humanas complejas como el texto escrito. Esto se
ha utilizado, por ejemplo, para realizar traducción automática
\cite{stahlberg2020neural}, detección de lenguaje de odio
\cite{macavaney2019hate} y producción automática de noticias
\cite{popel2020news}. Asimismo las mismas técnicas han sido utilizadas para
modelar otra producción humana con fines comunicacionales: la música
\cite{briot2020deep}. Entre los logros más salientes se encuentran generar un
agente que improvisa a la par de un pianista \cite{benetatos2020bachduet},
generar composiciones con estructuras jerárquicas coherentes \cite{Berardinis}
y componer automáticamente continuaciones a partir de un extracto musical pero
decidiendo el estilo de la composición de forma independiente
\cite{payne2019musenet}. En este trabajo proponemos generar y evaluar un modelo  
para la tarea de \emph{transferencia de estilo}, donde se busca tomar un
fragmento musical y reinterpretarlo en un estilo musical distinto al original.

En el área de composición automática de música se persiguen diversos objetivos.
Algunos trabajos buscan simplemente poder generar composiciones, normalmente
con restricciones de estilo o instrumentación. Por ejemplo,
\cite{sun2015ragtime} compone melodías de RagTime, \cite{hadjeres2017bach}
produce melodías y acompañamientos de corales de Bach y \cite{eck2002first}
presenta secuencias de acordes de estilo Blues. En otros casos, se busca tener
control más detallado sobre las composiciones a partir de información
contextual para el modelo. Ejemplos de esto es la composición de patrones de
batería a partir de una línea de bajo \cite{makris2017bass} o composición de
melodías en estilo pop a partir de una secuencia de acordes
\cite{yang2017midinet}. Finalmente, algunos sistemas buscan poder controlar el
estilo de la composición, como MuseNet \cite{payne2019musenet} o DeepJ
\cite{mao2018deepj}.

Este trabajo se enfoca en la tarea de \emph{transferencia de
estilo}, donde se busca reinterpretar un fragmento musical dado en un estilo
particular. Este problema fue abordado en el ámbito de imágenes, donde es
posible generar una imagen que tenga el contenido de una imagen original y el
estilo de una imagen de referencia. Esto se ha realizado, por ejemplo, con
pinturas \cite{gatys2015style} y con fotos \cite{luan2017photo}. Otra tarea
asociada es la modificación de un estímulo para que adquiera características
particulares. Por ejemplo, en imágenes es posible agregarle anteojos o bello
facial a la foto de una cara, así como envejecerla \cite{upchurch2016feature}.
Este tipo de transformaciones se han llevado a cabo mediante manipulaciones del
espacio latente generado por redes neuronales que aprenden a codificar
instancias del dominio de trabajo en espacios de baja dimensionalidad. Las
arquitecturas comunmente utilizadas para este propósito son los autoencoders
\cite{hou2016feature}.  

Esta técnica también se ha utilizado para manipular
fragmentos musicales. Por ejemplo, MusicVAE \cite{roberts2019musicvae} permite
generar nuevos fragmentos musicales como la interpolación entre otros dos
fragmentos. Otro ejemplo es el trabajo de Guo et al. \cite{guo2020variational},
donde esta técnica se utiliza para modificar atributos musicales específicos,
como el nivel de tensión tonal. En el presente trabajo proponemos basarnos en
estas técnicas para proponer y evaluar un modelo que permita hacer
transferencia de estilo para fragmentos musicales.

Las tareas de generación automática de estímulos musicales no pueden ser
evaluadas de forma supervisada como otras tareas de inteligencia artificial
debido a que se están generando ejemplos nuevos. Es por eso que la evaluación
suele consistir de comparaciones entre modelos \cite{Berardinis}, evaluaciones
donde personas deben juzgar las nuevas producciones \cite{hadjeres2017bach,
gatys2015style, luan2017photo} o
métricas que implementan juicios humanos para evaluar correctitud para la
tarea \cite{benetatos2020bachduet}. Debido a que en la revisión bibliográfica 
realizada no encontramos trabajos que aborden la tarea de transferencia de
estilos, como parte del trabajo tomaremos el enfoque de
\cite{benetatos2020bachduet} y propondremos métricas de evaluación para la
tarea.

\section*{Actividades y metodologías}
El presente plan de tesis consiste de las siguientes tareas. Luego del listado
se presenta más detalle respecto de las mismas.

\begin{itemize}
\item Interiorización y aprendizaje de las técnicas utilizadas en el área
  mediante la lectura de un libro de técnicas de aprendizaje profundo
  (Goodfellow et al. 2016 \cite{goodfellow2016deep}) y otro de técnicas de
  composición automática (Briot et al. 2020 \cite{briot2020deep}).
\item Entendimiento de los datasets existentes de música simbólica  \cite{Cuthbert2010Music21AT}.
\item Desarrollo de una propuesta de arquitectura para la tarea deseada \cite{guo2020variational, roberts2019musicvae}.
\item Preparación del pipeline de preprocesamiento del dataset para la arquitectura propuesta \cite{Cuthbert2010Music21AT}.
\item Entrenamiento de modelos en base a la arquitectura y diferentes datasets o preprocesamientos.
\item Desarrollo de propuestas de evaluación de las producciones realizadas.
\item Implementación y ejecución de las evaluaciones.
\item Escritura de la tesis.
\end{itemize}

\subsection*{Técnicas}
El problema de transferencia de estilos ha tenido su auge en los últimos años
con el desarrollo de nuevas técnicas de aprendizaje profundo. En este
sentido, ya que en el problema particular de la música no se han propuesto
técnicas específicas, surge la propuesta de basarse en técnicas aplicadas a
imágenes \cite{briot2020deep}. En particular, una de estas aproximaciones es
con el uso de Autoencoders \cite{goodfellow2016deep}. Estos modelos aprenden a
codificar un dominio específico en un espacio de dimensionaldad reducida,
conocido como el espacio latente, para luego decodificarlo. A partir de esto,
es posible codificar un estímulo al espacio latente, realizar modificaciones
en este espacio y luego decodificarlo para obtener una versión modificada del
estímulo original. Por ejemplo, eso fue realizado en imágenes de caras para
agregar o remover atributos a la misma (por ejemplo, agregar lentes o
envejecerlo) \cite{upchurch2016feature}.

En el dominio de la música, MusicVAE \cite{roberts2019musicvae} es un ejemplo
de un modelo basado en autoencoders utilizado para composición automática. En
este caso, el mismo fue utilizado para modificar atributos musicales de un
fragmento musical (por ejemplo la densidad de notas o el nivel de 
síncopa), así como realizar interpolaciones entre dos fragmentos 
musicales. Otro ejemplo de uso de autoencoders en música es el de Guo et al. 
\cite{guo2020variational}, donde la manipulación del espacio latente generado
por un autoencoder es utilizada para modificar el nivel de tensión de un
fragmento musical. En este trabajo proponemos basarnos en este último modelo,
entrenarlo sobre un conjunto de datos de fragmentos musicales de distintos
estilos y realizar modificaciones del espacio latente generado para
reinterpretar los fragementos en un estilo distinto al original.

\subsection*{Datasets}
Como parte del proyecto se debe buscar un conjunto de datos que permitan
entrenar y evaluar el modelo para la tarea. El conjunto de datos debe cumplir
con dos características: representación simbólica de los ejemplos musicales y
contener ejemplos diferenciados en varias categorías. La representación
simbólica refiere a la definición de qué notas suenan y cuando. Esto es en
contraste con datasets de audio, donde se representa el sonido final. El
requisito de un conjunto de datos de piezas musicales en representación
simbólica se debe a que la mayoría de los modelos computacionales de
composición automática trabajan con esta representación.  La necesidad de tener
los fragmentos diferenciados en categorías se debe a que buscamos conocer cómo
se representa cada categoría en el espacio latente y poder realizar la
reinterpretación de un fragmento de una categoría en otra.  Entre posibles
fuentes de datos se encuentran: MAESTRO, Lakh MIDI, Kunstderfuge, 8notes, MWD,
Classical Archive, Nottingham y KernScores.

Por otra parte, la arquitectura elegida (\cite{guo2020variational})
necesitará un formato particular de los datos. En particular, se necesitan como
entrada fragmentos de largo fijo con exactamente 2 pistas (voces o
instrumentos) cada uno. Esto significa que, en caso de que el fragmento musical
tenga más de dos pistas, será necesario convertirlas. Se considera utilizar  
herramientas ya desarrolladas para este fin presentadas por los autores de la
arquitectura en la que nos vamos a basar (\texttt{midi-miner}
\cite{Guo-reduccion}).

\subsection*{Evaluación}
Para evaluar la \textit{performance} de los modelos para esta tarea no hay aún
nada estandarizado como métrica cuantitativa puesto que la tarea en sí no está
muy desarrollada en la actualidad \cite{briot2020deep}. 
Debido a esto, como parte de este trabajo desarrollaremos métricas de
evaluación objetiva de los fragmentos creados por el modelo.  Por lo tanto, se
plantean 3 preguntas iniciales para las cuales definiremos métricas de
evaluación. Si llamamos $m$ al fragmento musical inicial y $m'$ al fragmento
convertido en un nuevo estilo, querremos evaluar: 

\begin{itemize}
    \item Musicalidad: el fragmento $m'$, ¿suena \textit{musical}?
    \item Estilo: este fragmento, ¿suena reinterpretado al estilo musical
      objetivo?
    \item Conservación de la canción original: ¿se puede observar que $m'$ 
    está relacionado con $m$? Dicho de otra manera, ¿$m’$ es parecido a $m$?
\end{itemize}

Para evaluar la musicalidad de un nuevo fragmento musical proponemos el uso del
la métrica Information Rate \cite{lattner2018},
métrica basada en la Teoría de la Información \cite{shannon1948} que ha sido
usada para este fin en distintos anteriores \cite{lattner2018, WangDubnov2015}.
Para comparar el estilo del fragmento original y el convertido, proponemos
utilizar la caracterización de estilo musical presentada en \cite{zivic2013},
donde se analiza la similitud estílistica de composiciones musicales en formato
simbólico. Finalmente, para evaluar si el fragmento convertido se asemeja al
fragmento original, buscaremos compararlos entre si con métricas de distancia
en el espacio de representación simbólico.

\section*{Plazos y factibilidad}

Este plan de tesis se deriva del trabajo realizado por el estudiante durante
una Beca de Iniciación a la Investigación en Ciencias de la Computación
(BIICC). El tesista participó del programa durante el año 2021 con los
directores como mentor e investigador responsable. 
Durante el plazo de la misma, el estudiante llevó a cabo varias de las tareas
definidas en este plan. En particular, el estudiante se interiorizó en la
temática mediante las lecturas de los libros y papers de referencia, investigó
posibles datasets a utilizar y se eligió un conjunto de datos, investigó
posibles implementaciones de modelos para elegir uno específico y escribió el
código que permite el proceso de entrenamiento y evaluación del modelo. Además
se definieron métricas preliminales para la evaluación de la tarea.

Como resultado del trabajo de la pasantía, se presenta este plan que busca
extender el trabajo realizado en una tesis de la carrera. Considerando los
avances del estudiante tanto en la comprensión del tema como en la
implementación de los modelos y la métrica de evaluación, se considera que la
misma puede ser finalizada en los plazos estipulados por el reglamento. El
trabajo restante incluye la implementación final de las métricas de evaluación,
la recopilación y análisis de los resultados de evaluación del modelo propuesto
y la presentación del trabajo realizado en formato de tesis. Se espera que el
mismo pueda ser presentado para defender a mediados del año 2023.

%\tripleSignature{Alumno (Lucas Somacal)}{Director (Martín Miguel)}{Director (Diego Fernández Slezak)}

% \section{Bibliografía}

\bibliographystyle{unsrt}
{\small
\bibliography{referencias}
}

\end{document}
